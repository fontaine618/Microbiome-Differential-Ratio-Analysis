\documentclass[english]{article}
\usepackage[T1]{fontenc}
\usepackage[latin9]{inputenc}
\usepackage[
paperwidth=210mm,
paperheight=297mm,
left=50pt,
top=50pt,
textwidth=345pt,
marginparsep=25pt,
marginparwidth=124pt,
textheight=692pt,
footskip=50pt
]{geometry}

\usepackage{float}
\usepackage{booktabs}
\usepackage{amsmath}
\usepackage{amssymb}
\usepackage{bbm}
\usepackage{mathabx}

\newcommand{\eps}{{\varepsilon}}

% bold symbols
\newcommand{\bx}{{\mathbf x}}
\newcommand{\bz}{{\mathbf z}}
\newcommand{\bt}{{\mathbf t}}
\newcommand{\by}{{\mathbf y}}
\newcommand{\bl}{{\mathbf l}}
\newcommand{\bu}{{\mathbf u}}
\newcommand{\br}{{\mathbf r}}
\newcommand{\bk}{{\mathbf k}}
\newcommand{\bm}{{\mathbf m}}
\newcommand{\bp}{{\mathbf p}}
\newcommand{\bj}{{\mathbf j}}
\newcommand{\ba}{{\mathbf a}}
\newcommand{\bw}{{\mathbf w}}
\newcommand{\bs}{{\mathbf s}}
\newcommand{\bb}{{\mathbf b}}
\newcommand{\be}{{\mathbf e}}
\newcommand{\bd}{{\mathbf d}}
\newcommand{\bg}{{\mathbf g}}

\newcommand{\bE}{{\mathbf E}}
\newcommand{\bX}{{\mathbf X}}
\newcommand{\bS}{{\mathbf S}}
\newcommand{\bZ}{{\mathbf Z}}
\newcommand{\bA}{{\mathbf A}}
\newcommand{\bD}{{\mathbf D}}
\newcommand{\bM}{{\mathbf M}}
\newcommand{\bC}{{\mathbf C}}
\newcommand{\bG}{{\mathbf G}}
\newcommand{\bI}{{\mathbf I}}
\newcommand{\bB}{{\mathbf B}}
\newcommand{\bP}{{\mathbf P}}
\newcommand{\bJ}{{\mathbf J}}
\newcommand{\bR}{{\mathbf R}}
\newcommand{\bU}{{\mathbf U}}
\newcommand{\bT}{{\mathbf T}}
\newcommand{\bH}{{\mathbf H}}
\newcommand{\bY}{{\mathbf Y}}
\newcommand{\bW}{{\mathbf W}}
\newcommand{\bL}{{\mathbf L}}
\newcommand{\bK}{{\mathbf K}}
\newcommand{\bQ}{{\mathbf Q}}

\newcommand{\balpha}{{\boldsymbol \alpha}}
\newcommand{\bbeta}{{\boldsymbol \beta}}
\newcommand{\bmu}{{\boldsymbol \mu}}
\newcommand{\bTheta}{{\boldsymbol \Theta}}
\newcommand{\bsigma}{{\boldsymbol \sigma}}
\newcommand{\btheta}{{\boldsymbol \theta}}
\newcommand{\bdelta}{{\boldsymbol \delta}}
\newcommand{\bxi}{{\boldsymbol \xi}}
\newcommand{\bphi}{{\boldsymbol \phi}}
\newcommand{\bpsi}{{\boldsymbol \psi}}
\newcommand{\btau}{{\boldsymbol \tau}}
\newcommand{\bzeta}{{\boldsymbol \zeta}}
\newcommand{\bepsilon}{{\boldsymbol \epsilon}}
\newcommand{\beps}{{\boldsymbol \varepsilon}}
\newcommand{\bgamma}{{\boldsymbol \gamma}}
\newcommand{\bnu}{{\boldsymbol \nu}}
\newcommand{\bPhi}{{\boldsymbol \Phi}}
\newcommand{\bSigma}{{\boldsymbol \Sigma}}
\newcommand{\blambda}{{\boldsymbol \lambda}}
\newcommand{\bpi}{{\boldsymbol \pi}}

\newcommand{\bone}{{\mathbf 1}}
\newcommand{\bzero}{{\mathbf 0}}

% blackboard symbols
\newcommand{\bbR}{{\mathbb R}}
\newcommand{\bbE}{{\mathbb E}}
\newcommand{\bbN}{{\mathbb N}}
\newcommand{\bbP}{{\mathbb P}}
\newcommand{\ind}{{\mathbbm 1}}

% calligraphic symbols
\newcommand{\cA}{\mathcal A}
\newcommand{\cB}{\mathcal B}
\newcommand{\cD}{\mathcal D}
\newcommand{\cE}{\mathcal E}
\newcommand{\cG}{\mathcal G}
\newcommand{\cH}{\mathcal H}
\newcommand{\cI}{\mathcal I}
\newcommand{\cL}{\mathcal L}
\newcommand{\cM}{\mathcal M}
\newcommand{\cN}{\mathcal N}
\newcommand{\cO}{\mathcal O}
\newcommand{\cP}{\mathcal P}
\newcommand{\cQ}{\mathcal Q}
\newcommand{\cS}{\mathcal S}
\newcommand{\cV}{\mathcal V}
\newcommand{\cW}{\mathcal W}
\newcommand{\cT}{\mathcal T}



% math operators
\DeclareMathOperator{\NN}{NN}
\DeclareMathOperator{\Var}{Var}
\DeclareMathOperator{\Cov}{Cov}
\DeclareMathOperator{\logit}{logit}
\DeclareMathOperator{\ELBO}{ELBO}
\DeclareMathOperator{\KL}{KL}
\DeclareMathOperator{\diag}{diag}
\DeclareMathOperator{\prox}{prox}
\DeclareMathOperator{\BIC}{BIC}
\DeclareMathOperator{\EBIC}{EBIC}
\DeclareMathOperator{\AIC}{AIC}
\DeclareMathOperator{\Tr}{Tr}
\DeclareMathOperator{\clr}{clr}
\DeclareMathOperator{\sgn}{sgn}
\DeclareMathOperator{\tvec}{vec}
\DeclareMathOperator{\ESS}{ESS}
\DeclareMathOperator{\df}{df}
\DeclareMathOperator{\RSS}{RSS}

% distributions
\DeclareMathOperator{\Bernoulli}{Bernoulli}
\DeclareMathOperator{\Multinomial}{Multinomial}
\DeclareMathOperator{\Hypergeometric}{Hypergeometric}
\DeclareMathOperator{\Binomial}{Binomial}
\DeclareMathOperator{\Poisson}{Poisson}
\DeclareMathOperator{\tGamma}{Gamma}
\DeclareMathOperator{\InvGamma}{InvGamma}
\DeclareMathOperator{\tBeta}{Beta}
\DeclareMathOperator{\ZIP}{ZIP}
\DeclareMathOperator{\ZINB}{ZINB}
\DeclareMathOperator{\NB}{NegativeBinomial}
\DeclareMathOperator{\ED}{ED}
\DeclareMathOperator{\Tw}{Tweedie}
\DeclareMathOperator{\QP}{QuasiPoisson}
\DeclareMathOperator{\Dirichlet}{Dirichlet}
\DeclareMathOperator{\DP}{DP}
\DeclareMathOperator{\GEM}{GEM}
\DeclareMathOperator{\Categorical}{Categorical}
\DeclareMathOperator{\Uniform}{Uniform}
\DeclareMathOperator{\iid}{iid}

\newcommand{\GP}{\mathcal {GP}}
\newcommand{\TGP}{\mathcal {TGP}}
\newcommand{\SMGP}{\mathcal {SMGP}}
\newcommand{\SMTGP}{\mathcal {SMTGP}}
\newcommand{\WP}{\mathcal {WP}}
\newcommand{\IWP}{\mathcal {IWP}}
\newcommand{\MN}{\mathcal {MN}}

% differential operator
\makeatletter \renewcommand\d[1]{\ensuremath{%
  \;\mathrm{d}#1\@ifnextchar\d{\!}{}}}
\makeatother
\title{Microbiome Differential Ratio Analysis -- Notes}

% --------------------------------------------------------------------------------------
% Comments
\usepackage{todonotes}
\newcommand{\mukai}[1]{\todo[color=green!20, size=\small]{[MW] #1}}
\newcommand{\simon}[1]{\todo[color=orange!20, size=\small]{[SF] #1}}
\newcommand{\gen}[1]{\todo[color=blue!20, size=\small]{[GL] #1}}
\newcommand{\hui}[1]{\todo[color=purple!20, size=\small]{[HJ] #1}}
% --------------------------------------------------------------------------------------


\begin{document}

\maketitle

\simon{Example comment from Simon}
\mukai{Example comment from Mukai}
\gen{Example comment from Gen}
\hui{Example comment from Hui}

% ======================================================================================
\section{Problem description \& Notation}
% ======================================================================================

\begin{table}[h]
    \centering
    \begin{tabular}{rlp{8cm}}
        \toprule
        \textbf{Symbol} & \textbf{Domain} & \textbf{Description} \\
        \midrule
        $i$ & $1,\ldots, n$ & Subject index \\ 
        $j,k$ & $1, \ldots, p$ & Taxa index \\ 
        $\bx_i$ & $\bbR^{p}$ & Taxa count/proportion vector for subject $i$\\ 
        $p_{jk}$ & $[0,1]$ & $p$-value for a differential ratio test between taxa $j$ and $k$ \\ 
        \bottomrule
    \end{tabular}
\end{table}



% ======================================================================================
\section{Motivation: Beta-Uniform mixture}
% ======================================================================================

Suppose we perform all $p(p-1)/2$ pairwise tests between taxa $j$ and taxa $k$. Under the null hypothesis, we
expect the $p$-value for the test to be uniformly distributed between 0 and 1. For non-null hypothesis, we expect
the $p$-value to have a non-uniform distribution over the unit interval and that this distribution is skewed
towards 0. A common modeling assumption for the non-null distribution is $\tBeta(a, 1)$. Indeed, this 
is commonly known as a Beta-Uniform Mixture (BUM) and it is often used to model a collection of $p$-values
containing some nulls and some non-nulls.

Hence, for a taxa $j$, we can consider all $p-1$ other taxa and model the collection of $p$-values accordingly
\begin{align*}
    p_{jk} \sim \pi_j \underbrace{\Uniform[0,1]}_{\text{nulls}} +
    (1-\pi_j) \underbrace{\tBeta(a_j, 1)}_{\text{non-nulls}},\qquad k\neq j
\end{align*}
We can borrow from this decomposition to propose a model for each pairwise test.

% ======================================================================================
\section{Strength-corrected Stochastic Block Model for Weighted Networks}
% ======================================================================================

What we know/expect:
\begin{itemize}
    \item If taxa $j$ and $k$ are DA, then the pairwise test could be non-null (it is possible the change is the same)
    \item If taxa $j$ is DA, but not $k$, then the pairwise test should be non-null
    \item If both $j$ and $k$ are not DA, then the pairwise test should be null
\end{itemize}
At the population level, we thus get a standard SBM with:
\begin{itemize}
    \item Two communities, determined by DA vs non-DA
    \item All edges between non-DA and DA
    \item Most edges DA to DA
    \item No edges non-DA to non-DA
\end{itemize}
This resembles a core-periphery graph structure.

However, we rather observe the network of $p$-values. We consider the following version of SCWSBM.
Each node $j$ is assigned to a community $c_j\in [C]$. Each node $j$ also has a parameter $\theta_j\geqslant 0$,
the strength-correction. For each pair of communities $cd$, we define a model $f_{cd}$ that depends on some parameter
$a_{cd}$ and $\theta_j,\theta_k$ (provided $c_j=c$ and $c_k=d$).
\begin{align*}
    c_j &\sim \Categorical(\bpi) \\ 
    p_{jk} &\sim f_{c_jc_k} (\cdot\mid a_{c_jc_k}, \theta_j,\theta_k)
\end{align*}

In our case, we could define two communities indexed by 0 (nulls) and 1 (non-nulls). The models would be:
\begin{align*}
    f_{00}(p) &= \Uniform_{[0, 1]}(p)  & \text{(non-DA to non-DA)}\\
    f_{10}(p\mid a_{10}, \theta_1, \theta_0) &= \tBeta(p\mid a_{10}\theta_1\theta_0, 1) & \text{(DA to non-DA)}\\
    f_{11}(p\mid a_{11}, \theta_1, \theta_2) &= \tBeta(p\mid a_{11}\theta_1\theta_2, 1) & \text{(DA to DA)}\\
\end{align*}\simon{Not sure if it should be $\tBeta(a,1)$ or $\tBeta(1,b)$?}
with
\begin{align*}
    \theta_j \geqslant 1, a_{cd} \geqslant 1
\end{align*}
This condition ensures the correct interpretation that the components corresponds to non-nulls.


Comments:
\begin{itemize}
    \item The strength-correction $\theta_j$ can be understood as the tendency to produce $p$-values
    close to 0 when non-null. This might be associated with the variability of the taxa, or its density.
    For example, we expect to have less power for rare taxa, which could transpire into a smaller $\theta_j$.
    \item We could start with a simpler approach without strength correction: estimation
    should be simpler/faster/more stable. Eventually we could compare with the more flexible 
    model to see if we gain anything. I think it would be interesting to compare the
    estimated $\theta_j$ to properties of the counts (sparsity, variability, etc.)
    \item This formulation does not fully capture the expected behavior, since the DA to DA model
    does not allow for nulls. One option is to do nothing and expect $a_{11}<a_{10}$. Otherwise,
    we could further choose a BUM for that component
\end{itemize}


% ======================================================================================
\section{Generative model}
% ======================================================================================

The above method models the edge-level $p$-values directly. I wonder if we could describe a 
generative model that generates this structure starting from the node level instead? 
Here, we probably
need more assumptions about what DA means and how it translates to the pairwise test. 
In particular, we probably need to be careful about what the null hypotheses are. 
This exercise may also help to extract what are the sufficient conditions on the $p$-values
for this approach to work \& make sense.

% ======================================================================================
\section{EM algorithm}
% ======================================================================================

The joint distribution is given by
\begin{align*}
    p(\bP, \bz\mid \ba, \btheta, \bpi)
    = 
    \prod_{j=1}^{p}  \pi_{c_j}
    \times \prod_{j<k} f_{c_jc_k}(p_{jk}\mid a_{c_jc_k}, \theta_j, \theta_k)
\end{align*}
We can write it in a longer form
\begin{align*}
    p(\bP, \bz\mid \ba, \btheta, \bpi)
    = 
    \prod_{c=0}^{1}\prod_{j=1}^{p}  \pi_{c}^{\ind[z_j=c]}
    \times \prod_{c,d=0}^1\prod_{j<k} [f_{cd}(p_{jk}\mid a_{cd}, \theta_j, \theta_k)]^{\ind[z_j=c]\ind[z_k=d]}
\end{align*}
so that the joint log-likelihood can be written as
\begin{align*}
    \ell(\ba, \btheta, \bpi) 
    =
    \sum_{c=0}^{1}\sum_{j=1}^{p} \ind[z_j=c]\log\pi_c
    + 
    \sum_{c,d=0}^{1}\sum_{j<k} 
    \ind[z_j=c]\ind[z_k=d]
    \log f_{cd}(p_{jk}\mid a_{cd}, \theta_j, \theta_k)
\end{align*}

For the E-step, we need to compute the posterior distribution for $Z_j$. In particular, we need 
$\bbP[Z_j=c]$ for all $c$ and $\bbP[Z_j=c, Z_k=d]$ for all $c,d$, so we can replace it in the expressions. 
However, the products
$\ind[z_j=c]\ind[z_k=d]$ are harder to deal with, so a Variational EM is used instead, where
we assume factorization across $j$ for the approximate posterior of $\bz$. 
\simon{It appears most SBM for weighted edges use a VEM approach.}
Hence, we parameterize the
approximate posterior by
\begin{align*}
    q(\bz\mid \btau) = \prod_{j=1}^{p} \prod_{c=0}^{1} \tau_{j,c}^{\ind[z_j=c]}
\end{align*}
with the constraints $\sum_{c=0}^{1}\tau_{j,c}=1$. This probably leads to a fixed point algorithm 
(See Mariadassou, 2010, Section 4.2). The M step probably looks like a few weighted MLE problems: since
the $\theta_j$ will be shared between the models, there will probably not be a closed form solution and
will require some gradient- or fixed-point approach.

\subsection{Variational inference}

For the E-step, we need to produce the $Q$ function, defined by the expected value of the joint 
log-likelihood, where the integrate the latent variables. Here, the latent variables are the 
community assignments $z_j$. Let $\langle \cdot\rangle^{(t)}$ denote the expectation wrt the current
(iteration $(t)$) values of the global parameters $\bphi=(\ba, \btheta, \bpi)$, that is,
\begin{align*}
    \langle f(\bz)\rangle^{(t)}
    = 
    \bbE\left\{ f(\bz) \mid \ba^{(t)}, \btheta^{(t)}, \bpi^{(t)}\right\}
\end{align*}
The $Q$ function is then given by
\begin{align*}
    Q(\bphi\mid\bphi^{(t)})
    &=
    \left\langle \ell(\ba,\btheta,\bpi) \right\rangle^{(t)} \\ 
    &=
    \sum_{c=0}^{1}\sum_{j=1}^{p}\left\langle \ind[z_j=c]\right\rangle^{(t)}\log\pi_c \\
    &\quad + 
    \sum_{c,d=0}^{1}\sum_{j<k} 
    \left\langle\ind[z_j=c]\ind[z_k=d]\right\rangle^{(t)}
    \log f_{cd}(p_{jk}\mid a_{cd}, \theta_j, \theta_k)
\end{align*}
Hence, the E-step requires the computation of posterior community assignment probability and the joint
assignment probabilities. The joint assignment probabilities are much more complicated to compute,
so a Variational EM is used instead.

In variational inference, instead of working with the true posterior distribution, as is required by
the standard EM algorithm, we use an approximate posterior instead. We choose an approximate posterior
among a simpler family of distribution according to some criterion. In particular, we choose
\begin{align*}
    q(\bz\mid\btau^{(t)} ) =  \prod_{j=1}^{p} \prod_{c=0}^{1} \tau_{j,c}^{(t)\ind[z_j=c]}
    \approx p(\bz\mid \bphi^{(t)})
\end{align*}
with $\tau_{j,c}^{(t)}\geqslant 0$ and $\sum_{c=0}^{1}\tau_{j,c}^{(t)}=1$, and 
where the variational parameters $\btau^{(t)}$ are chosen to minimize the variational objective:
\begin{align*}
    \ELBO(\tau, \bphi)
    &= 
    \underbrace{\log p(\bP\mid \bphi)}_{\text{marginal log-likelihood}} 
    - \underbrace{\KL(q(\cdot\mid\tau) \Vert p(\cdot\mid \bphi))}_{\text{post. approx. error}} \\ 
    &= 
    \underbrace{\cH(q)}_{\text{entropy}} 
    + \underbrace{\bbE_q\left\{\log p(\bP, \bz\mid \bphi)\right\}}_{\text{exp. log joint likelihood}} 
\end{align*}
In standard EM, there is no approximation so the posterior approximation error is 0 and the VEM
objective is simply the marginal likelihood: this is the same objective as EM. When an approximation
is used, the the KL is positive and the ELBO becomes a lower bound to the evidence (marginal
log-likelihood).


Under the factorized form chosen for the approximate posterior, the last expectation is easier to 
compute than the $Q$ function above. Indeed, by the independence across nodes, we get that the joint
assignment probabilities equals the product of the marginals:
\begin{align*}
\bbE_q\{\ind[z_j=c]\} = \tau_{j,c},\qquad 
    \bbE_q\{\ind[z_j=c]\ind[z_k=d] \} = \tau_{j,c}\tau_{k,d}
\end{align*}
Then, we need to compute the entropy of $q$:
\begin{align*}
    \cH(q)
    &= 
    -\bbE_q\{\log q(\bz\mid \btau)\} 
    = -\sum_{j=1}^{p}\sum_{c=0}^{1} \tau_{jc}\log\tau_{jc}
\end{align*}
Hence, the evidence lower bound (ELBO) is given by
\begin{align*}
     \ELBO(\tau, \bphi)
    &=
    -\sum_{j=1}^{p}\sum_{c=0}^{1} \tau_{jc}\log\tau_{jc} +
    \sum_{j=1}^{p}\sum_{c=0}^{1} \tau_{jc}\log\pi_{c} \\
    &\quad +
    \sum_{c,d=0}^{1}\sum_{j<k} \tau_{jc}\tau_{kd}\log f_{cd}(p_{jk}\mid a_{cd}, \theta_j, \theta_k)
\end{align*}
Similar to the VEM algorithm, the VEM algorithm alternates between the posterior and the global parameters:
\begin{description}
    \item[VE-step] Update the posterior approximation by solving
    \begin{align*}
        \btau^{(t+1)} = \underset{\btau}{\arg\max } \ELBO(\btau, \bphi^{(t)})
    \end{align*}
    \item[M-step] Update global parameters
    \begin{align*}
        \bphi^{(t+1)} = \underset{\bphi}{\arg\max}\ELBO(\btau^{(t+1)}, \bphi)
    \end{align*}
\end{description}

\subsection{VE-step}

As in Mariadassou (2010), a fixed point algorithm can be used. We need to remember that the $\btau$'s must
satisfy simplex constraints within each node. We can first compute the derivatives:
\begin{align*}
    \frac{\partial}{\partial\tau_{jc}}
    \ELBO(\tau,\bphi)
    =
    -\log\tau_{jc} - 1 +\log\pi_c + 
    \sum_{k\neq j} \sum_{d=0}^{1} \tau_{kd}\log  f_{cd}(p_{jk}\mid a_{cd}, \theta_c,\theta_d)
\end{align*}
For the sum-to-one constraint, we include a Lagrange multiplier for each node:
\begin{align*}
    \ELBO(\btau,\bphi) + \sum_{j=1}^{p} \lambda_i (\btau_j^\top\bone - 1 )
\end{align*}
Taking derivatives yields
\begin{align*}
     \frac{\partial}{\partial\tau_{jc}}
    \left(\sum_{j=1}^{p} \lambda_i (\btau_j^\top\bone - 1 )\right)
    = \lambda_j
\end{align*}
We thus find the equation
\begin{align*}
    0 = 
    -\log\tau_{jc} - 1 +\log\pi_c + 
    \sum_{k\neq j} \sum_{d=0}^{1} \tau_{kd} \log f_{cd}(p_{jk}\mid a_{cd}, \theta_c,\theta_d)
    + \lambda_j
\end{align*}
Hence, we get the implicit equation
\begin{align*}
    \tau_{jc} = \pi_c e^{\lambda_j-1} \prod_{k\neq j}\prod_{d=0}^{1} 
    f_{cd}(p_{jk}\mid a_{cd}, \theta_c,\theta_d)^{\tau_{kd}}
\end{align*}
Thus, as long as $\pi_c\geqslant 0$, we get the positive constraint for $\btau$ for free. 
The $e^{\lambda_j-1}$ term can absorb the sum-to-one constraint so that we get
\begin{align*}
    \tau_{jc} \propto \pi_c \prod_{k\neq j}\prod_{d=0}^{1} 
    f_{cd}(p_{jk}\mid a_{cd}, \theta_c,\theta_d)^{\tau_{kd}}
\end{align*} 

\subsection{M-step}

\subsubsection{Prior community probabilities}

Similarly, we add a Lagrange multiplier for the sum-to-one constraint $\sum_c\pi_c=1$:
the derivative wrt $\pi_c$ is given by
\begin{align*}
    \frac{\partial}{\partial\pi_c} \ELBO(\btau,\bphi)
    &= 
    \pi_c^{-1}\sum_{j=1}^{p}\tau_{jc}
\end{align*}
With the Lagrange multiplier, we get the equations
\begin{align*}
    \lambda\pi_c = \sum_{j=1}^{p} \tau_{jc}
\end{align*}
From the sum-to-one constraint, we get
\begin{align*}
    \pi_c \propto \sum_{j=1}^{p} \tau_{jc}
\end{align*}

The update for $\ba$ and $\btheta$ depend on the chosen models. Without strength correction,
there should be analytical updates. With Sc, it probably is better to perform gradient updates
until convergence (some re-parameterization might be helpful to encode constraints.) For example,
we could define $a_{cd} = 1+\exp(b_{cd})$ and compute the derivative wrt $b_{cd}$.

\subsection{Community rates: no strength correction case}

In this scenario, the only parameter left is $\ba$. We need to assume $a_{cd}\geqslant 1$.
First, $a_{00}=1$ is fixed to get a uniform distribution. Then, we are left with two parameters
(by symmetry): both have the same update, so we focus on $a_{10}$. The first-order condition is given by
\begin{align*}
    0 = \sum_{j<k} (\tau_{j0}\tau_{k1} + \tau_{j1}\tau_{k0})
    \frac{\frac{\partial}{\partial a_{10}}f_{10}(p_{jk}\mid a_{10})}{f_{10}(p_{jk}\mid a_{10})}
\end{align*}
\simon{to be continued ... }
% ======================================================================================
\section{Comments}
% ======================================================================================

\begin{itemize}
    \item To decide null vs non-null, it might be better to use a specific threshold 
    other than 0.5: the posterior community membership probability could be used 
    similar to a $p$-value (there might 
    be some sort of empirical Bayes or Bayes factor argument there)
    \item This framework completely abstract the pairwise test: anything  
    (parametric, non-parametric)could be used for the pairwise test
    provided the $p$-values satisfy some properties yet TBD
    \item This might be generalized to any problem with the following structure: we want node-level inference by aggregating edge-level tests
    \item We could consider an increasing sequence of complexity (maybe some complexity is not necessary):
    \begin{enumerate}
        \item SBM with thresholded $p$-values
        \item SCSBM with thresholded $p$-values
        \item WSBM 
        \item SCWSBM
    \end{enumerate}
    \item Then, we could study theoretical properties
    \begin{itemize}
        \item Community label accuracy will be related to Power and FDR
        \item Posterior probability of labels could be used as some sort of node $p$-value?
        \item The variational approximation might have the desirable effect of
        producing independent posterior probabilities, so common FDR-correction 
        methods could be applicable?
    \end{itemize}
\end{itemize}


\end{document}
